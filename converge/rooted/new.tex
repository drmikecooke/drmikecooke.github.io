%% LyX 2.3.2 created this file.  For more info, see http://www.lyx.org/.
%% Do not edit unless you really know what you are doing.
\documentclass{article}
\usepackage[latin9]{inputenc}
\usepackage{amsmath}
\usepackage[unicode=true]
 {hyperref}
\begin{document}
C{o{n{v{e{r{g{e{n{c{e}}}}}}}}}}

\href{http://5.html}{◀ Previous}

\hypertarget{rooted-6}{

\section{Rooted 6}

\label{rooted-6}}

\hypertarget{axiomatic-binding}{

\subsubsection{Axiomatic binding}

\label{axiomatic-binding}}

I have layed out three possible ways to create real numbers from rationals
(decimals, Dedekind, Cauchy). There are other ways, which I am only
vaguely aware of from statements in books and papers. Although they
seem different, they are found to be essentially the same through
the ``axiomatic method''. The way it works is to find a set of features
(``axioms'') that the constructions share, and then prove that any
systems that share these features are mathematically the same, or
``isomorphic'' as it's called in the trade. The choice of axioms
is based on a number of criteria that include taste, minimality, obviousness,
simplicity, and so on. The criteria clash to some extent, and the
choice is often dictated by history and hence the mood of the time
of their creation, along with evolution and simplifications according
to prevailing fashion.

Most text books settle on preserving the mathematical features of
the rational numbers in terms of addition, multiplication, order,
learned in secondary school. These features can be axiomatized but
they are not ``categorical'', which means that they do not lead
to the systems that satisfy them being isomorphic. In fact, there
is a whole range of ``fields'', as these systems are called (in
physics this word refers to something very different), between the
rationals and the reals. If the ordering requirements are dropped,
there are further fields, such as the complex numbers and ones with
just a finite number of elements (``Galois fields''). Relaxing the
multiplication axioms, so that for some elements $ab\neq ba$, expands
the possibilities to include ``quaternions''. . . .

The important addition for reals, as already mentioned, is the least
upper bound property. The constuctions show the existence of systems
that satisfy the proposed axioms (given only in sketch form here).
Uniqueness is shown by constucting isomorphisms between any two systems.

Firstly, to give just a sketch of what is done, there are disingished
elements, ``0'' and ``1'', in any field. It either is proved,
or stated, that these elements are unique in each particular system,
so they must be uniquely related under the isomorphism. The ``natural
numbers'', $\mathbb{N}=\{0,1,2,3,\dots\}$, can be constructed uniquely
in each real system from addition of the 1s. The sequence never terminates
due to the order properties, $n<n+1$, since $0\lt1$ and $x\lt y\implies n+x\lt n+y$.
In similar ways, the negative integers $\mathbb{Z}^{-}$ and rationals
$\mathbb{Q}$ can be uniquely related.

The capstone is to relate the irrationals using the least upper bound
property. In any particular system, the set of rationals less than
a given element has a corresponding set of rationals in the other:
$D(r)=\{q\in\mathbb{Q}:q\lt r\}\rightarrow\{q'\}\subset\mathbb{Q}'$.
The least upper bound of the source set is $r$, and there is a unique
least upper bound of the target set, which we can label $r'$, such
that $r\rightarrow r'$, completeing the isomorphism.

A final point that may or not worry the reader is how the rationals
seem to be categorical, but their characterization as an ordered field
does not lead to this. In fact, there is an added condition that makes
the rationals categorical --- they are the \emph{minimal} ordered
field or ``prime field'' contained in $\mathbb{R}$. 
\end{document}
