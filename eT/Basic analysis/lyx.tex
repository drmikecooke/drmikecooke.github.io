%% LyX 2.3.2 created this file.  For more info, see http://www.lyx.org/.
%% Do not edit unless you really know what you are doing.
\documentclass[english]{article}
\usepackage[T1]{fontenc}
\usepackage[latin9]{inputenc}

\makeatletter

%%%%%%%%%%%%%%%%%%%%%%%%%%%%%% LyX specific LaTeX commands.
%% Binom macro for standard LaTeX users
\newcommand{\binom}[2]{{#1 \choose #2}}


%%%%%%%%%%%%%%%%%%%%%%%%%%%%%% User specified LaTeX commands.
\usepackage{braket}

\makeatother

\usepackage{babel}
\begin{document}
We want to show the sequence in:

\[
\lim_{N\to+\infty}e_{N}=\lim_{N\to+\infty}\left(1+\frac{1}{N}\right)^{N}
\]

\noindent is monotonic increasing for $N>0$. As a bonus we want to
also show that $\lim_{N\to-\infty}e_{N}$ is monotonic decreasing
for $N<-1$, along with $e_{-N}>e_{M}$ for all positive $N,M$. Finally
we want to show that both limits go the same finite number. The numbers
$N=0,-1$ are pathological, giving $e_{0}=+\infty$ and $e_{-1}=0$.

None of these statements is obvious (except for the exceptions) first
off: for example, the size of the factors $(1+1/N)$ decrease as $N\to+\infty$,
but there are more of them. We can bring to bear stuff we ``know''
from school or university, but then we have to ask ``How do we know?''
Some of this ``knowledge'' includes features of the behaviour of
powers (or logarithms) that we want to prove.

We can use the binomial formula to break the problem down a bit:

\[
\left(a+b\right)^{N}=\sum_{n=0}^{N}a^{n}b^{N-n}\binom{N}{n}
\]

The binomial factor is:

\[
\binom{N}{n}=\frac{N!}{n!\left(N-n\right)!}
\]

\noindent is the number of ways of choosing $n$ $a$s and $N-n$
$b$s from the brackets. {[}$n!=n(n-1)(n-2)\ldots1.${]} This can
be shown fairly easily using inductive arguments, and elementary properties
of multiplication and addition.

So we have (swapping $n$ and $N-n$):

\[
e_{N}=\sum_{n=0}^{N}\left(\frac{1}{N}\right)^{n}1^{N-n}\binom{N}{n}=\sum_{n=0}^{N}\left(\frac{1}{N}\right)^{n}\binom{N}{n}
\]

The first two terms turn out to be both 1, so they are constant for
$N\to+\infty$. The third term is $\left(N-1\right)/\left(2N\right)=\left(1/2\right)\left(1-1/N\right)$,
which approaches 1/2 from below (i.e. is monotonic increasing). Although
this is a good start, obviously to consider each term in this way
would take from here to eternity. Let us define:

\[
e_{N}^{\left(n\right)}=\left(\frac{1}{N}\right)^{n}\binom{N}{n}
\]

Separating out the binomial factor:

\[
e_{N}^{\left(n\right)}=\frac{1}{n!}\prod_{m=0}^{n}\frac{N-m}{N}=\frac{1}{n!}\prod_{m=1}^{n}\frac{N-m}{N}=\frac{1}{n!}\prod_{m=1}^{n}\left(1-\frac{m}{N}\right)
\]

This is a product of terms that increase towards 1. The m=0 term was
removed since it was (already) 1. Hence the $e_{N}$ increase with
$N$.

Unfortunately we can't transfer this analysis straight away to negative
$N$ since the binomial expansion is infinite (and not at this stage
obviously true). However we can cast the negative values to positive
using:

\[
e_{-N}=\left(1-\frac{1}{N}\right)^{-N}=\left(\frac{N-1}{N}\right)^{-N}=\left(\frac{N}{N-1}\right)^{N}=\frac{N}{N-1}\left(1+\frac{1}{N-1}\right)^{N-1}=\frac{N}{N-1}e_{N-1}
\]

While interesting, this result isn't much use at this point, although
it does tell us that $e_{-N}>e_{N-1}$. Let's look instead at:

\[
f_{N}=\frac{1}{e_{-N}}=\left(1-\frac{1}{N}\right)^{N}=\sum_{n=0}^{N}\left(\frac{-1}{N}\right)^{n}\binom{N}{n}=1-1+\frac{N-1}{2N}-\frac{\left(N-1\right)\left(N-2\right)}{6N^{2}}+\ldots=\frac{N^{2}-1}{3N^{2}}+\ldots
\]

If we can show that this increases with $N$ we will then have that
$e_{N}$ decreases. We have paired the positive and negative terms
to give zero and a number that increases to 1/3 as $N$ increases,
which is promising for the project.

Working in more generalised terms:

\[
f_{2N+1}=\sum_{n=0}^{N}\left(\left(\frac{-1}{2N+1}\right)^{2n}\binom{2N+1}{2n}+\left(\frac{-1}{2N+1}\right)^{2n+1}\binom{2N+1}{2n+1}\right)
\]

After some hard mathematical graft, I compressed this a bit to:

\[
f_{2N+1}=\sum_{n=0}^{N}\left(2n\left(\frac{1}{2N+1}\right)^{2n+1}\binom{2N+2}{2n+1}\right)
\]

You may have noticed that I have missed out the even $N$. In that
case there are an odd number of terms. Adding in the extra (positive)
term gives:

\[
f_{2N}=\sum_{n=0}^{N-1}\left(2n\left(\frac{1}{2N}\right)^{2n+1}\binom{2N+1}{2n+1}\right)+\left(\frac{1}{2N}\right)^{2N}
\]

We also notice that the $n=0$ term is zero, and can therefore be
omitted from consideration. Let us define:

\[
f_{N}^{(n)}=2n\left(\frac{1}{N}\right)^{2n+1}\binom{N+1}{2n+1}
\]

These expressions mean that:

\[
f_{2N}=\sum_{n=1}^{N-1}f_{2N}^{(n)}+\left(\frac{1}{2N}\right)^{2N}
\]

\[
f_{2N+1}=\sum_{n=1}^{N}f_{2N+1}^{(n)}
\]

Let us unpack:

\[
f_{N}^{(n)}=\frac{2n}{\left(2n+1\right)!}\left(\frac{1}{N}\right)^{2n+1}\prod_{m=0}^{2n}\left(N+1-m\right)=\frac{2n}{\left(2n+1\right)!}\prod_{m=0}^{2n}\frac{N+1-m}{N}
\]

All the terms in the final product increase with $N$ except for the
$m=0,1$ terms. However the $m=0$ term can be combined with the $m=2$
term, while the $m=1$ term is 1, neither increasing or decreasing
the product:

\[
f_{N}^{(n)}=\frac{2n}{\left(2n+1\right)!}\frac{N^{2}-1}{N^{2}}\prod_{m=3}^{2n}\frac{N+1-m}{N}
\]

The combined $m=0,2$ term is also increasing with $N$.

The only niggle left is whether the combination of the last two terms
might decrease relative to the previous single term. However, one
finds that in fact the worry about the behaviour of the final terms
can be lifted by considering:

\[
f_{2N}^{(N)}=2N\left(\frac{1}{2N}\right)^{2N+1}\binom{2N+1}{2N+1}=\left(\frac{1}{2N}\right)^{2N}
\]

So in fact:

\[
f_{2N}=\sum_{n=1}^{N}f_{2N}^{(n)}
\]

\noindent and the neighbouring series are comparable, including the
last odd term, if present.

So what do we know so far? First, for $N,m>0$ we have $e_{N+m}>e_{N}$.
Similarly, $e_{-\left(N+m\right)}<e_{-N}$ ($N>1$). As noted above
$e_{-N}>e_{N-1}$. We can now extend this to $e_{-N}>e_{-\left(N+m\right)}>e_{N+m-1}$.
The $e_{N}$ is an increasing sequence, this means that all $e_{-N}$
are greater than all $e_{M}$, or $e_{-N}>e_{M}$ ($M>0$). Any e\_\{-N\}
bounds above the increasing monotonic sequence e\_M. Similarly any
$e_{M}$ bounds the decreasing sequence $e_{-N}$ from below. The
other bound is given by the first terms in the sequences. These features
enable us to say that the sequences tend towards unique numbers (rather
than increasing or decreasing forever, wandering off to positive or
negative infinity). Although the sequences are made up of rational
numbers, the limits may not be (and in fact are not). The real numbers
were invented to enable such sequences to be dealt with coherently.
Further the limits of the sequences lead to the \textit{same} number.
This follows from (since the extra factor tends to 1):

\[
e_{-N}=\frac{N}{N-1}e_{N-1}
\]

There are a bewildering number of ways to define real numbers from
the rationals, the two main ones being Cauchy sequences and Dedekind
cuts. These eventually lead to things like the ``bounded monotone
covergence theorem'' used above.
\end{document}
